
\chapter{Introduction}\label{Cap:Introduction}
	
	\section{READ ME FIRST!}
	This template \LaTeX was prepared by the Internationalization Committee of the Post Graduate Program in Electrical Engineering (PPGEE) of the University of Brasilia, with the purpose of providing guidelines to students with dissertations or thesis works in progress.
	
	Under the internal regulations of PPGEE, Doctoral dissertations must be written in English. The master's thesis should be written in English, but Portuguese may be acceptable. Note that the use of terms `dissertation' and `thesis' in English are switched if compared to the ideas conveyed in Portuguese language. This is not an error; it is just how the English language works. If in doubt, refer to \href{https://www.enago.com/academy/thesis-vs-dissertation/}{this page}.
	
	This document has been adjusted to comply with Brazilian ABNT standards for academic writing, whilst keeping English as the default language. Observe that captions in the cover page, signatures page and catalog record page are supposed to be kept in Portuguese. The same happens with the section `Resumo'. Even if the chosen idiom is English, the Portuguese version must be provided.
	
	Citations follow the ABNT standard, for instance \cite{Saadat1999,ABNT1993}. You can use $\backslash$citeonline, as spoke \citeonline{MatthewN.O.Sadiku2000}, or maybe someone else.
	
	Table titles are placed at the top, sources at the bottom. \textbf{It is mandatory to specify the sources (references)} for all tables, even if they were compiled by the author - in such cases, a note `Own authorship' is sufficient.
		\begin{table}[H]
		\renewcommand{\arraystretch}{1.3}
		\caption{Relative permitivitty ($\varepsilon_r$) and electrical resistivity ($\rho$) of soils and common materials }
		\label{table:soil_properties}
		\centering
		\begin{tabular}{|c|c|c|c|c|c|}
			\hline
			\textbf{Dry materials} & \boldmath{$\varepsilon_r$} & \boldmath{$\rho$ [$\Omega$.m]} & 
			\textbf{Saturated materials} &
			\boldmath{$\varepsilon_r$} &
			\boldmath{$\rho$ [$\Omega$.m]} \\
			\hline
			Air & 1 & $10^9$--$10^{15}$ & Distilled water & 81 & $10^5$ \\ \hline
			Sand and gravel & 2--6 & $10^5$ & Fresh water & 81 & 2000 \\ \hline
			Clay & 5 & 300--5000 & Sea water & 81 & <10 \\ \hline
			Shale and dry silt & 5 & 1000 & Sand & 20--30 & 1000--$10^4$ \\ \hline
			Limestone gravel & 4 & $7\times10^6$ & Silt & 10 & 100--1000 \\ \hline
			Sandy soil & 2.6 & 1000--8000 & Clay & 40 & <10 \\ \hline
			Loamy soil & 2.4 & 300--5000 & Sandy soil & 25 & <150 \\ \hline
			Granite & 5 & 1500--$10^4$ & Granite & 7 & 1000 \\ \hline
			Limestone & 4 & 500--5000 & Limestone & 8 & 500 \\ \hline
			Salt & 5--6 & 1000--$10^5$ & Loamy soil & 15 & 20 \\ \hline
			Granite gravel & 5 & $1.5\times10^6$--$4.5\times10^{6}$ & Granite gravel & 7 & 5000--$10^4$ \\ \hline
			Basalt & 6 & 1000 & Silt & 30 & 10 \\ \hline
			Diabase & 7 & 100 & Shale & 7 & 10 \\ \hline
			Iron & 1 & $9.70\times10^{-8}$ & Limestone gravel & 8 & 2000--3000 \\ \hline
			Carbon steel & 1 & $1.43\times10^{-7}$ & Diabase & 8 & 10 \\ \hline 
			PVC & 8 & $15\times10^{17}$ & Basalt & 8 & 100 \\ \hline
			Asphalt & 3--5 & $2\times10^{6}$--$30\times10^{6}$ & Asphalt & 3--5 & $10^4$--$6\times10^{6}$ \\ \hline
			Dry concrete & 5.5 & $10^6$--$10^{9}$ & Wet concrete & 12.5 & 21--100 \\
			\hline
		\end{tabular}
		\vspace{0.05cm}\\
		\textbf{Source:} ~Own authorship.\\
	\end{table}
	
	The same happens with the figures: titles are placed at the top, sources at the bottom. \textbf{It is mandatory to specify the sources (references)} for all figures and graphs, even if they were compiled by the author - in such cases, a note `Own authorship' is sufficient.
	\begin{figure}[H]
		\begin{center}
		\caption{Just a sample figure.} 
			\includegraphics[width=.1\columnwidth]{./CapaUNB_Nova.eps}
				\vspace{0.05cm}\\
		\textbf{Source:} ~Own authorship.\\
				\label{fig:sample_figure}
			\end{center}
		\end{figure}

    Text encoding has been set to UTF-8. This is important to allow displaying accented characters in Portuguese without loss of compatibility with the \textit{babel} English package. Note that UTF-8 encoding is set in line 8 of the main \LaTeX file, as well as hard-coded in line 19 of the \textit{unbthesis.cls} class file. Be careful when messing around with character encoding to prevent conflicts and compiling issues.
    
    This is a constant work in progress and new improvements are expected over time. Instructions and responses to FAQs will be added to this section as need. You can contact the main author of this template via email: \href{mailto:amaurigm@lapse.unb.br}{amaurigm@lapse.unb.br}.