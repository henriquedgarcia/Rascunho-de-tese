% Pacotes de linguagem
\usepackage[english,brazil]{babel}
\usepackage[T1]{fontenc}
\usepackage[utf8]{inputenc}
\usepackage{lmodern}

% Pacotes de matemática, fontes matemáticas, icones matemáticos, etc.
\usepackage{amssymb}    %  Adiciona símbolos matemáticos extras
\usepackage{amsfonts}   %  Fornece fontes matemáticas adicionais
\usepackage{amstext}    %  Permite inserir texto normal dentro de expressões matemáticas  (\text{})
\usepackage{amsmath}    % Expande os recursos matemáticos do LaTeX com ambientes extras (align, multiline, etc)
\usepackage{mathtools}  % extensão do amsmath que melhora a tipografia matemática
\usepackage{calc}       % Permite expressões aritméticas em comandos como \setlength.

%% Formatação de texto e espaçamento
\usepackage{blindtext}    % Gera texto fictício (lorem ipsum) para testes.
\usepackage{setspace}     % Controla o espaçamento entre linhas (simples, 1.5, duplo).
\usepackage{verbatim}     % Permite exibir texto exatamente como digitado, útil para código.
\usepackage{indentfirst}  % Faz com que o primeiro parágrafo após um título seja indentado.
\usepackage{textcomp}     % Oferece símbolos adicionais como °, ©, ™, entre outros.
\usepackage{soul}         % Adiciona efeitos como sublinhado, tachado e espaçamento entre letras.
\usepackage{xfrac}        % Cria frações em linha com barra inclinada (ex: ½ como `\sfrac{1}{2}`).
\usepackage{icomma}  % Corrige o espaçamento após vírgulas em modo matemático
\usepackage{paralist}  % Cria listas compactas (compactitem, compactenum) e listas inline (inparaenum).
\usepackage{relsize}  % Ajusta tamanho de fonte relativo ao atual (\larger, \smaller).
\usepackage{enumitem}
\usepackage{afterpage}  % Executa comandos após a quebra de página (ex: \afterpage{\clearpage}).
\usepackage{fancyvrb}  % Melhora ambientes verbatim (numeração de linhas, molduras, cores).
\usepackage[retainorgcmds]{IEEEtrantools} % Permite usar comandos do estilo IEEE (\IEEEPARstart, \IEEEeqnarray) em outras classes.

%% Cores e estilo
\usepackage{color}     % Permite usar cores no texto e elementos
\usepackage[bookmarks=true,bookmarksopen=false,colorlinks=true,linktocpage=true,pagebackref	]{hyperref}  % Cria links clicáveis e configurações de navegação em PDF.
\usepackage{url}  % Formata URLs corretamente, com quebras de linha seguras.
\usepackage{lscape}  % Gira o conteúdo da página para modo paisagem.
\usepackage{hhline} % pacote para criar linhas horizontais segmentadas
\usepackage{layout}  % Exibe graficamente o layout da página atual (margens, espaçamentos, etc).
\usepackage{fancyhdr}  % Personaliza cabeçalhos e rodapés.


%% Tabelas
\usepackage{stfloats}  % Melhora o posicionamento de figuras e tabelas em documentos com duas colunas.
\usepackage{longtable}  % Cria tabelas que podem se estender por várias páginas.
\usepackage{multirow}   % Permite mesclar células verticalmente em tabelas.
\usepackage{hhline}     % Cria linhas horizontais personalizadas em tabelas.
\usepackage{array}      % Oferece mais controle sobre colunas em tabelas.
\usepackage{booktabs}   % Cria tabelas com aparência profissional (sem linhas verticais).
\usepackage{ctable}  % Gera tabelas e figuras com legenda e notas em um único comando.
\usepackage{colortbl}  % Adiciona cores em células de tabelas.


% Imagens e figuras
\usepackage{subfigure}  % Cria subfiguras dentro de uma figura maior (embora esteja obsoleto; `subcaption` é recomendado).
\usepackage{float}      % Permite posicionar figuras e tabelas com mais precisão (ex: `[H]`).
\usepackage{graphics,graphicx}  % Inserção e manipulação de imagens.
\usepackage{epstopdf}  % Converte imagens EPS para PDF automaticamente.
\usepackage{psfrag}  % Substitui texto em arquivos EPS por construções LaTeX.
\usepackage{wallpaper}  % Insere imagens de fundo (inclusive em modo mosaico).

%% Citações e referências
\usepackage[alf,abnt-etal-text=it,abnt-etal-list=0,abnt-etal-cite=2,abnt-and-type=&,abnt-year-extra-label=yes]{abntex2cite}  % Formata citações e referências conforme normas da ABNT.

%% Algoritmos
\usepackage{algorithm}      % Cria ambientes flutuantes para algoritmos.
\usepackage{algpseudocode}  % Permite escrever pseudocódigo com comandos como `\If`, `\While`, etc.
