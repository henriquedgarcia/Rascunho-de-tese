
\chapter{Foundations}\label{Cap:Foundations}

\section{Transmissão de vídeos 360 com ladrilhos}

O streaming de vídeo 360 envolve várias etapas de processamento de imagens, como mostra a figura XXXX. O processo começa com a captura do vídeo através de arranjos de câmeras tradicionais. As imagens precisam ser processadas, coladas e então projetadas em uma superfície plana onde um codificador de vídeo como H.265 ou AV1 é usado para comprimi-lo. Como o campo visual do ser humano é limitado, o usuário encherga apenas uma fração de toda as esfera. Assim, para que o que é visto tenha uma boa resolução, como Full DH, a projeção deve ter uma resolução muito maior, como 4K ou até mesmo 12K. Porém, como processar partes da esfera que não são assistidas desperdiçam recursos, a projeção é segmentada espacialmente em ladrilhos (tiles) que possam ser decodificados forma independentes.

A

A criação do streaming de vídeo esférico envolve várias etapas envolvendo diversas técnicas de processamento de imagem para

Captura -> stitch --> projeção --> tiling --> codificação -->  Dashing --> Streaming --> decoding --> mount --> Rectilinear Projection (Viewport) --> display

\subsection{Captura}
\subsection{Projeção}
\subsection{Tiling}
\subsection{codificação}
\subsection{Dashing}
\subsection{Viewport}
\subsection{Limitações da Literatura}
\subsection{Conexão como trabalho proposto}
\subsection{Avaliação de Desempenho}


\section{360EAVP}
\subsection{Preparação}
\subsection{Identificador de face}
\subsection{Preditor de Viewport}
\subsection{Sobre a estrutura do programa}

\section{Modelagem de QoE}
