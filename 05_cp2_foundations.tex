
\chapter{FUNDAMENTAÇÃO TEÓRICA E TRABALHOS RELACIONADOS}\label{Cap:Foundations}

\section{Transmissão de vídeos 360 com ladrilhos}

O streaming de vídeo 360 envolve várias etapas de processamento de imagens, como mostra a figura XXXX. O processo começa com a captura do vídeo através de arranjos de câmeras tradicionais. As imagens precisam ser processadas, coladas e então projetadas em uma superfície plana onde um codificador de vídeo como H.265 ou AV1 é usado para comprimi-lo. Como o campo visual do ser humano é limitado, o usuário encherga apenas uma fração de toda as esfera. Assim, para que o que é visto tenha uma boa resolução, como Full DH, a projeção deve ter uma resolução muito maior, como 4K ou até mesmo 12K. Porém, como processar partes da esfera que não são assistidas desperdiçam recursos, a projeção é segmentada espacialmente em ladrilhos (tiles) que possam ser decodificados forma independentes.

A criação do streaming de vídeo esférico envolve várias etapas envolvendo diversas técnicas de processamento de imagem para

Captura -> stitch --> projeção --> tiling --> codificação -->  Dashing --> Streaming --> decoding --> mount --> Rectilinear Projection (Viewport) --> display

\subsection{Captura}

\begin{figure}[tbh]
	\centering
	\includegraphics[width=0.4\linewidth]{img/captura1}
	\caption{legenda aqui}
	\label{fig:captura1}
\end{figure}

\begin{figure}[tbh]
	\centering
	\includegraphics[width=0.4\linewidth]{img/registração}
	\caption{legenda aqui}
	\label{fig:registracao}
\end{figure}

\begin{figure}[tbh]
	\centering
	\includegraphics[width=0.4\linewidth]{img/eyefish}
	\caption{legenda aqui}
	\label{fig:eyefish}
\end{figure}

\begin{figure}[tbh]
	\centering
	\includegraphics[width=0.4\linewidth]{img/esfera}
	\caption{legenda aqui}
	\label{fig:esfera}
\end{figure}


\begin{figure}[tbh]
	\centering
	\includegraphics[width=0.7\linewidth]{"img/360 Video - building"}
	\caption{legenda aqui}
	\label{fig:building_360_video}
\end{figure}






\subsection{Projeção}
\href{https://wiki.panotools.org/Cubic_Projection}{https://wiki.panotools.org/Cubic\_Projection}

\subsubsection{Projeção Cubica Rectilinear}
\href{https://wiki.panotools.org/Panorama_Viewers}{https://wiki.panotools.org/Panorama\_Viewers}

\begin{figure}[tbh]
	\centering
	\includegraphics[width=0.7\linewidth]{"img/projecao_cmp"}
	\caption{legenda aqui}
	\label{fig:projecao_cmp}
\end{figure}

\subsubsection{Projeção Cubica RectilinearProjeção Equirretangular (Projeção cilíndrica equidistante ou projeção de Plate Carré)}

\begin{itemize}
	\item \url{https://www.infoescola.com/cartografia/projecao-cilindrica-equidistante/}
	\item \url{https://en.wikipedia.org/wiki/Equirectangular_projection}
	\item \url{https://pt.wikipedia.org/wiki/Proje%C3%A7%C3%A3o_cil%C3%ADndrica}
	\item \url{https://pt.wikipedia.org/wiki/Proje%C3%A7%C3%A3o_cil%C3%ADndrica_equidistante}
	\item \url{https://docs.qgis.org/3.40/pt_BR/docs/gentle_gis_introduction/coordinate_reference_systems.html}
	\item \url{https://www.fcav.unesp.br/Home/departamentos/engenhariarural/TERESACRISTINATARLEPISSARRA/edital.pdf}
	\item \url{https://brasilescola.uol.com.br/geografia/projecoes-cartograficas.htm}
	\item \url{https://www.infoescola.com/cartografia/projecao-cilindrica-equidistante/}
	\item \url{https://proj.org/en/stable/operations/projections/eqc.html}	
\end{itemize}

Os prints abaixo são desse livro

\url{https://www.google.com.br/books/edition/Flattening_the_Earth/0UzjTJ4w9yEC?hl=pt-BR&gbpv=1&dq=isbn:0226767477&printsec=frontcover}

\begin{figure}[tbh]
	\centering
	\includegraphics[width=0.7\linewidth]{"img/screenshot_livro1"}
	\caption{legenda aqui}
	\label{fig:screenshot_livro1}
\end{figure}

\begin{figure}[tbh]
	\centering
	\includegraphics[width=0.7\linewidth]{"img/screenshot_livro2"}
	\caption{legenda aqui}
	\label{fig:screenshot_livro2}
\end{figure}

\begin{figure}[tbh]
	\centering
	\includegraphics[width=0.7\linewidth]{"img/tissot"}
	\caption{Indicador de tissot para a projeção cubemap}
	\label{fig:tissot}
\end{figure}



\subsection{Tiling}
\subsection{codificação}
\subsection{Dashing}
\subsection{Viewport}
\subsection{Limitações da Literatura}
\subsection{Conexão como trabalho proposto}
\subsection{Avaliação de Desempenho}


\section{360EAVP}
\subsection{Preparação}
\subsection{Identificador de face}
\subsection{Preditor de Viewport}
\subsection{Sobre a estrutura do programa}

\section{Modelagem de QoE}
