\documentclass{unbthesis}
\usepackage{amssymb,amsfonts,amstext,amsmath}
\DeclareMathOperator{\sen}{sen}
\DeclareMathOperator{\argum}{arg}
\usepackage[english,brazil]{babel}
\usepackage{blindtext}
\usepackage[T1]{fontenc}
\usepackage[utf8]{inputenc}
\usepackage{color}
\usepackage{subfigure}
\usepackage{setspace}
\usepackage{longtable}
\usepackage{float}
\usepackage{colortbl}
\usepackage{verbatim}
\usepackage{lscape}
\usepackage{indentfirst}
\usepackage[bookmarks=true,bookmarksopen=false,linktocpage=true,pagebackref]{hyperref}
\usepackage[alf,abnt-etal-text=it,abnt-etal-list=0,abnt-etal-cite=2,abnt-and-type=&,abnt-year-extra-label=yes]{abntex2cite}

\usepackage{algorithmic}
\usepackage{textcomp}
\usepackage{stfloats}
\usepackage{xfrac}
\usepackage{booktabs}
\usepackage{soul}
\usepackage{subfigure}
\usepackage{multirow}
\usepackage{hhline}
\usepackage{graphics,graphicx}
\usepackage{epstopdf}
\usepackage{array}
\usepackage{icomma}

% Configurações do pacote backref
% Usado sem a opção hyperpageref de backref
\renewcommand{\backrefpagesname}{Cited in page(s):~}
% Texto padrão antes do número das páginas
\renewcommand{\backref}{}
% Define os textos da citação
\renewcommand*{\backrefalt}[4]{
\ifcase #1 %
No citation in text.%
\or
Cited in page #2.%
\else
Cited #1 times in pages #2.%
\fi}%


\usepackage{booktabs}
\usepackage{ctable}
\usepackage{multirow}
\usepackage{fancyhdr}
\usepackage{fancyvrb}
\usepackage{relsize}
\usepackage{calc}
\usepackage{psfrag}
\usepackage{epstopdf}
\usepackage{wallpaper}
\usepackage{paralist}
%\usepackage{hhline} %pacote para criar linhas horizontais segmentadas
%\usepackage{url}
%\usepackage{layout}
%\usepackage[retainorgcmds]{IEEEtrantools}
\usepackage{afterpage}

\setlength{\LTcapwidth}{17cm} \setlength{\textwidth}{17cm}
\setlength{\textheight}{24.9cm} \setlength{\topmargin}{-0.5in}

%NOVOS COMANDOS
\newcommand{\FuncSen}[1]{\mbox{\hspace{0.5mm}sen\hspace{0.5mm}}#1}
\newcommand{\FuncCos}[1]{\mbox{\hspace{0.5mm}cos\hspace{0.5mm}}#1}
\newcommand{\palfa}[1]{\mbox{plano-$\alpha$}#1}

\usepackage{mathtools}
\DeclarePairedDelimiter\abs{\lvert}{\rvert}%

%DEIXA ESPAÇOS EXTRAS, QUE O LATEX CRIA, NO FIM DA PÁGINA
%\raggedbottom



\addto\captionsbrazil{%
	\renewcommand{\contentsname}{Table of Contents}
	\renewcommand{\listfigurename}{List of Figures}
	\renewcommand{\listtablename}{List of Tables}
	\renewcommand{\refname}{References}
	\renewcommand{\bibname}{References}
	\renewcommand{\chaptername}{Chapter}
	\renewcommand{\figurename}{Figure}
	\renewcommand{\tablename}{Table}
	\renewcommand{\acknowledgementsname}{Acknowledgements}
}


%%%VARIÁVEIS DO TEMPLATE - AJUSTAR DE ACORDO - se usar direito vai ter menos trabalho!

%variáveis exibidas na capa e contracapa
\newcommand{\ttitle}{Rascunho de tese: A dataset for performance assessments of ABR algorithms on tiled 360° video streaming}
\newcommand{\tauthor}{Henrique Domingues Garcia}
%\newcommand{\tdoctype}{Exame de Qualificação de Tese de Doutorado}
\newcommand{\tdoctype}{Tese de Doutorado}
\newcommand{\tcourse}{Engenharia Elétrica}
\newcommand{\tdepartment}{Departamento de Engenharia Elétrica}
\newcommand{\tdpt}{ENE} %ou ENC ENM etc

\newcommand{\tdegree}{Doutor}
% ou \newcommand{\tdegree}{Doutor}

%variáveis utilizadas na ficha catalográfica
\newcommand{\tlocal}{Brasília/DF}
\newcommand{\tday}{03}
\newcommand{\tmonth}{setembro}
\newcommand{\tyear}{2022}

%nome para citação
\newcommand{\tsurname}{Garcia}
\newcommand{\tfirstname}{Henrique D.}



\begin{document}

%PEDIDO DE MODELO COM AS DIMENSÕES DA PÁGINA
%PRECISA DO PACKAGE "layout"
%\layout

%NOVO ESPAÇAMENTO
\baselineskip 20pt
\parskip 5pt
\pagestyle{empty}
\setlength{\headheight}{14.49998pt}

\pdfbookmark[0]{Cover}{Cover}
%=======================================================================
% Folha de título da dissertação
\ThisTileWallPaper{\paperwidth}{\paperheight}{./CapaUNB_Nova.eps}

%AJUSTANDO A DISTÂNCIA ENTRE O HEAD DA PÁGINA E O CORPO DO TEXTO
\setlength{\headsep}{10.2cm}

\thispagestyle{empty}
\begin{center}

% INSERIR AQUI O TÍTULO DA DISSERTAÇÃO
{\LARGE{\bf \MakeUppercase{\ttitle}}}

\vspace{2cm}

% INSERIR AQUI O NOME DO AUTOR DA DISSERTAÇÃO
{\large {\bf \MakeUppercase{\tauthor}}}

\vspace{1cm}

% ADAPTAR DE ACORDO

{\large {\bf \MakeUppercase{\tdoctype} \\
			 \MakeUppercase{em \tcourse}\\
			}}

\vspace{1cm}

{\large {\bf \MakeUppercase{\tdepartment}\\
}}

\vspace{3.5cm}

\textcolor{black}{\LARGE{\bf FACULDADE DE TECNOLOGIA }}


\vspace{1cm}

\textcolor{black}{\huge {\bf UNIVERSIDADE DE BRASÍLIA }}

\end{center}

\pagebreak

%=======================================================================

%AJUSTANDO A DISTÂNCIA ENTRE O HEAD DA PÁGINA E O CORPO DO TEXTO
%\ThisTileWallPaper{\paperwidth}{\paperheight}{doc/folha_de_assinaturas.pdf}
%\setlength{\headsep}{0.95cm}
\setlength{\headsep}{0.5cm}
%
\begin{bf}
\begin{center}
{\normalsize Universidade de Brasília}\\
{\normalsize Faculdade de Tecnologia}\\
{\normalsize \tdepartment}\\

\vspace{7mm}

%% INSERIR O NOME DA DISSERTAÇAO/TESE
%REALISTIC MODELING OF POWER LINES FOR TRANSIENT ELECTROMAGNETIC INTERFERENCE STUDIES
{\large {\bf \ttitle\\}}

\vspace{5mm}
%
%% INSERIR NOME DO AUTOR
{\large \tauthor}
\end{center}
\end{bf}
\vspace{1mm}
\noindent{{\bf \MakeUppercase{\tdoctype} SUBMETIDA AO PROGRAMA DE PÓS-GRADUAÇÃO EM ENGENHARIA ELÉTRICA DA \\ UNIVERSIDADE DE BRASÍLIA COMO PARTE DOS REQUISITOS NECESSÁRIOS PARA A OBTENÇÃO DO GRAU DE \MakeUppercase{\tdegree}.\\}}
\vspace{1mm}

\noindent\MakeUppercase{{\bf Aprovada por:\\}}


\begin{bf}
\vspace{2.5mm}
\noindent\rule{150mm}{0.1mm}\\
{ Nome, Título (Instituição) }\\
{(Orientador)}

\vspace{2.5mm}
\noindent\rule{150mm}{0.1mm}\\
{ Nome, Título (Instituição) }\\
{(Co-orientador)}

\vspace{2.5mm}
\noindent\rule{150mm}{0.1mm}\\
{ Nome, Título (Instituição) }\\
{(Examinador Externo)}

\vspace{2.5mm}
\noindent\rule{150mm}{0.1mm}\\
{ Nome, Título (Instituição) }\\
{(Examinador Externo)}

\vspace{2.5mm}
\noindent\rule{150mm}{0.1mm}\\
{ Nome, Título (Instituição) }\\
{(Examinador Interno)}

%
%%LOCAL E DATA
\vspace{2.5mm}
\noindent {\tlocal, \tmonth ~de \tyear.}
\end{bf}
%
\pagebreak 
%AJUSTANDO A DISTÂNCIA ENTRE O HEAD DA PÁGINA E O CORPO DO TEXTO
\setlength{\headsep}{0.95cm}
%Comparação de fase, elementos incrementais de corrente, componente CC de decaimento exponencial, proteção no domínio do tempo.
\begin{titlepage}
{
	\flushleft{\textbf{FICHA CATALOGRÁFICA}}\\
	\vspace{3mm}
	\fbox{
		\parbox{\linewidth}{
        \baselineskip 20pt
        \parskip 5pt
	\MakeUppercase{\tsurname, \tfirstname}\\
	\ttitle. [\tlocal] \tyear.\\
	xxx, nnnp., 210 x 297 mm (\tdpt/FT/UnB, \tdegree, \tdoctype, \tyear).\\
	Universidade de Brasília, Faculdade de Tecnologia, \tdepartment.\\
	\tdepartment\\
	\begin{tabular}{lll}
	1. Keyword & \hspace{1cm} &  2. Keyword \hspace{1cm} \\
	3. Keyword & \hspace{1cm} & 4. Keyword \hspace{1cm} \\
	5. Keyword & \hspace{1cm} & 6. Keyword \hspace{1cm} \\
	7. Keyword & \hspace{1cm} & 8. Keyword \hspace{1cm} \\
	I. \tdpt/FT/UnB           &  \hspace{1cm}    &   II. Título (série)\hspace{1cm}
	\end{tabular}
	}}\\
%Comparação de fase, elementos incrementais de corrente, componente CC de decaimento exponencial, proteção no domínio do tempo.
	\vspace{5mm}
	\textbf{REFERÊNCIA BIBLIOGRÁFICA}\\
	\MakeUppercase{\tsurname, \tfirstname} (\tyear). \ttitle. \tdoctype, Publicação PPGEE.XXXXX/\tyear, \tdepartment, Universidade de Brasília, Brasília, DF, xxxxp.
	\\
	\vspace{5mm}
	\textbf{CESSÃO DE DIREITOS}\\
	AUTOR: \tfirstname ~\tsurname\\
	TÍTULO: \ttitle.\\
	GRAU: \tdegree~~~~~~~~~~~ANO: \tyear\\
	\vspace{5mm}
	É concedida à Universidade de Brasília permissão para reproduzir cópias deste \tdoctype ~e para emprestar ou vender tais cópias somente para propósitos acadêmicos e científicos. O autor reserva outros direitos de publicação e nenhuma parte deste trabalho pode ser reproduzida sem autorização por escrito do autor.
	\vspace{7mm}
		\flushleft{
		 \underline{~~~~~~~~~~~~~~~~~~~~~~~~~~~~~~~~~~~~~~~~~~~~~~~~~~~~~~~~~~~~~~~~~~~~~}\\
		\tfirstname ~\tsurname \\
    	Universidade de Brasília (UnB)\\
	    Campus Darcy Ribeiro\\
	    Faculdade de Tecnologia - FT\\
    	\tdepartment (\tdpt)\\
    	Brasília - DF 		
    	CEP 70919-970
		}	
}
\end{titlepage}

%NOVO ESPAÇAMENTO
\baselineskip 22pt
\pdfbookmark[0]{Dedication}{Dedication}
\begin{dedicatory}
À Clarisse.
\end{dedicatory}
\pdfbookmark[0]{Acknowledgements}{Acknowledgements}
\acknowledgements
Agradeço à família que ajuda e atrapalha, mas nunca me abandona.
\pdfbookmark[0]{Abstract}{Abstract}
\abstract
Regardless of the idiom chosen, the Abstract in English must be provided.\\

\begin{keywords}
Keywords go here.
\end{keywords}
\pdfbookmark[0]{Resumo}{Resumo}
\resumo

Mesmo que a Tese ou Dissertação seja redigida em inglês, é necessário fornecer o resumo em português.\\

\begin{keywords}
Escrever palavras-chave.
\end{keywords}
%% DEFINE CABEÇALHO DO SUMÁRIO E DAS LISTAS

%IMPRESSÃO SOMENTE FRENTE
\pagestyle{fancy}
\renewcommand{\chaptermark}[1]{\markboth{ #1}{}}
\fancyhf{}
\fancyhead[RE,RO]{\fontsize{9}{10}\thepage}
\fancyhead[LO,LE]{\fontsize{9}{10}\textit{\textsc{\nouppercase\leftmark}}}

%NOVO ESPAÇAMENTO
\baselineskip 14pt
\pagenumbering{roman}

%% Sumário (gerado automáticamente)
\phantomsection %comando para fazer os bookmarks funcionarem na seção abaixo
\addcontentsline{toc}{chapter}{Table of contents}
\tableofcontents

%NOVO ESPAÇAMENTO
\baselineskip 23pt

%% Lista de figuras (gerada automaticamente)
\clearpage
\phantomsection
\addcontentsline{toc}{chapter}{List of figures}
\listoffigures

%% Lista de tabelas (gerada automaticamente)
\clearpage
\phantomsection
\addcontentsline{toc}{chapter}{List of tables}
\listoftables

%% Lista de Simbolos (gerada manualmente)
\markboth{LIST OF SYMBOLS}{LIST OF SYMBOLS}
\clearpage
\phantomsection
\addcontentsline{toc}{chapter}{List of symbols}
\chapter*{List of Symbols}

\noindent
\pagestyle{fancy}

\begin{longtable}{l l l p{0.86\linewidth}}
$\varepsilon_r$ &   & Relative electric permittivity &[p.u.]\\\\

\end{longtable} 

%% Glossario (gerada manualmente)
\markboth{GLOSSARY}{GLOSSARY}
\clearpage
\phantomsection
\addcontentsline{toc}{chapter}{Glossary}
\chapter*{Glossary}

\noindent
\begin{longtable}{l l p{0.86\linewidth}}

3LPE 	&   & Three-layer polyethilene \\\\

\end{longtable}


\pagestyle{empty}
\pagenumbering{arabic}

%NOVO ESPAÇAMENTO
\baselineskip 23pt
\parskip 5pt

%IMPRESSÃO SOMENTE FRENTE
\pagestyle{fancy}
\renewcommand{\chaptermark}[1]{\markboth{\thechapter\hspace{2mm}--\hspace{2mm}#1}{}}
\renewcommand{\sectionmark}[1]{\markright{\thesection\hspace{2mm}--\hspace{2mm}#1}{}}
\fancyhf{}
\fancyhead[RE,RO]{\fontsize{9}{10}\textit{\textsc{\thepage}}}
\fancyhead[LE,LO]{\fontsize{9}{10}\textit{\textsc{\rightmark}}}
\renewcommand{\footrulewidth}{0.0pt} %Definindo a espessura de uma linha do rodapé


\chapter{Introdução}\label{Cap:Introduction}
% Apresente o tema geral de vídeos 360 e a crescente demanda por transmissão eficiente:
% Exemplo: "Com a crescente popularidade de conteúdos imersivos como vídeos 360°, garantir uma transmissão eficiente e de alta qualidade tornou-se um desafio crucial para sistemas de streaming."
% Explique a importância da segmentação espacial (ladrilhos) e temporal no contexto de vídeos 360:
% Exemplo: "A abordagem baseada em ladrilhos permite transmitir apenas partes do vídeo que estão no campo de visão do usuário, otimizando o uso da largura de banda."
% Introduza o uso de MPEG DASH e algoritmos ABR (Adaptação de Taxa de Bits) para lidar com as variações na qualidade da rede.

	
\section{Contextualização}

Com a popularização de dispositivos de realidade virtual e mista, a demanda por conteúdos imersivos também aumenta, e para os distribuidores de conteúdo tornou-se um desafio garantir uma transmissão eficiente e de alta qualidade. Os vídeos esféricos, aka Videos 360º apresentam um dos maiores consumos de largura de banda pois os mecanismos de transmissão tradicionais processam regiões do vídeo que não são vistas pelo usuário. Como o usuário fica posicionado no centro de uma esfera e o vídeo é projetado na casca interna desta esfera, o usuário apenas assistirá a porção de vídeo que seu campo de visão (FOV -Field Of View)permitir. Os óculos de realidade virtual o FOV geralmente possui valores entre 90° e 200° horizontal e entre 90° e 140° vertical 
\href{https://vr-compare.com/}{https://vr-compare.com/}

\begin{figure}[tbh]
	\centering
	\includegraphics[width=0.7\linewidth]{img/fov_sizes_oculus}
	\caption[legenda curta]{Legenda longa permite até quebra. sei lá o que pode fazer}
	\label{fig:fovsizesoculus}
\end{figure}

A segmentação espacial da esfera em ladrilhos e a transmissão de apenas os ladrilhos que o usuário vê usando algum protocolo de HTTP Adaptive Streaming (HAS) é uma das soluções mais promissoras. Essa abordagem cede controle ao cliente de selecionar a qualidade do vídeo cuja taxa de bits melhor se acomoda à largura de banda disponível, evitando eventos de rebbufering e travamentos na reprodução.

\begin{figure}[tbh]
	\centering
	\includegraphics[width=0.7\linewidth]{img/projecao_com_tiles_e_viewport}
	\caption{projeção com tiles e viewport}
	\label{fig:projecaocomtileseviewport}
\end{figure}

Porém, no caso dos vídeos esféricos o cliente precisa saber antecipadamente a largura de banda e os ladrilhos que serão vistos nos próximos segundos. Várias técnicas desde regressão linear e aprendizado por reforço já foram propostas, porém, a muitos destes estudos avaliam seus resultados

\section{Problema}

Os estudos comparam seus resultados com a situação ideal ou ingênua. Os estudos inventam situações próprias as vezes impraticáveis ou irreais. Os protocolos reais mais utilizados são do tipo HAS (HTTP adaptive Streaming)

\subsection{Sobre o tempo de decodificação}

De acordo com o artigo do Flare o tempo de decodificação de dispositivos móveis é quase 50\% do atraso total da exibição. Logo, é necessário conhecer sua distribuição para modelar corretamente o limiar do \textit{buffer} de reprodução (que deve conter \textit{chunks} suficiente para que a reprodução não seja interrompida até que cheguem novos \textit{chunks}).

\subsection{Fluxo de trabalho}

\begin{algorithm}
	\caption{Algoritmo para controle do buffer}\label{alg:buffer_control}
	\begin{algorithmic}[1]
		
	\While {TRUE}
		\If {buffer < buffer\_limiar}
			\While {buffer < buffer\_max}
				\State \textbf{Request} chunks
			\EndWhile
		\EndIf
	\EndWhile
	\end{algorithmic}
\end{algorithm}

Limiar deve ser no mínimo o tempo necessário para solicitar, receber, decodificar e exibir um novo \textit{chunk}. Este atraso de reprodução do \textit{chunk} precisa ser menor do que a diferença do tempo que será reproduzido ($T_i$) e o tempo atual ($T_0$) (tempo que falta para o \textit{chunk} ser exibido) ou a reprodução será interrompida por esvaziamento do \textit{buffer}:

\begin{math}
	T_{exibicao} = 2T_{propagacao} + T_{transmissao} + T_{decodificacao} \leq T_i - T_0
\end{math}

\subsection{Objetivo}
% Declare claramente o objetivo do artigo

\begin{itemize}
	\item Este trabalho propõe uma metodologia objetiva para avaliar o desempenho de algoritmos de adaptação de taxa de bits (ABR) e preditor de viewport (VP) em protocolos HAS (HTTP Adaptive Streaming, ex. DASH, Smooth Streaming, HLS) para transmissão de vídeos 360° segmentados espacialmente em ladrilhos.
	\item Este trabalho propõe uma metodologia objetiva para avaliar o desempenho de algoritmos de adaptação de taxa de bits (ABR) e preditor de viewport (VP) em protocolos HAS (Adaptive Streaming over HTTP, ex. DASH, Smooth Streaming, HLS) para transmissão de vídeos 360° segmentados espacialmente em ladrilhos.
	\item Uma vez que os protocolos ASH segmentam temporalmente o vídeo em chunks de mesma duração, construiremos uma base de dados de métricas dos chunks dos ladrilhos, como tempo de decodificação, taxa de bits e métricas de qualidade objetiva como SSIM, MSE, S-MSE e WS-MSE.
\end{itemize}

\subsection{Contribuições}
% Liste as principais contribuições do artigo

\begin{itemize}
	\item Uma base de dados de métricas para vídeos em diferentes qualidades, criada para suportar simulações.
	\item Uma metodologia unificada para análise de algoritmos ABR com múltiplos elementos sincronizados.
	\item Um estudo de caso que compara o desempenho de diferentes algoritmos sob várias condições de rede, com métricas como SSIM, MSE e taxa de bits.	
\end{itemize}

\subsection{Estrutura do artigo}
%Descreva como o artigo está organizado
% Sempre use aspas automáticas, pois aspas duplas " dá problema com o pacote babel.

``O restante deste artigo está organizado da seguinte forma: a Seção 2 apresenta os trabalhos relacionados; a Seção 3 detalha a metodologia proposta; a Seção 4 discute os resultados experimentais; e a Seção 5 conclui o artigo e sugere direções para trabalhos futuros.''


\chapter{FUNDAMENTAÇÃO TEÓRICA E TRABALHOS RELACIONADOS}\label{Cap:Foundations}

\section{Transmissão de vídeos 360 com ladrilhos}

O streaming de vídeo 360 envolve várias etapas de processamento de imagens, como mostra a figura XXXX. O processo começa com a captura do vídeo através de arranjos de câmeras tradicionais. As imagens precisam ser processadas, coladas e então projetadas em uma superfície plana onde um codificador de vídeo como H.265 ou AV1 é usado para comprimi-lo. Como o campo visual do ser humano é limitado, o usuário encherga apenas uma fração de toda as esfera. Assim, para que o que é visto tenha uma boa resolução, como Full DH, a projeção deve ter uma resolução muito maior, como 4K ou até mesmo 12K. Porém, como processar partes da esfera que não são assistidas desperdiçam recursos, a projeção é segmentada espacialmente em ladrilhos (tiles) que possam ser decodificados forma independentes.

A criação do streaming de vídeo esférico envolve várias etapas envolvendo diversas técnicas de processamento de imagem para

Captura -> stitch --> projeção --> tiling --> codificação -->  Dashing --> Streaming --> decoding --> mount --> Rectilinear Projection (Viewport) --> display

\subsection{Captura}

\begin{figure}[tbh]
	\centering
	\includegraphics[width=0.4\linewidth]{img/captura1}
	\caption{legenda aqui}
	\label{fig:captura1}
\end{figure}

\begin{figure}[tbh]
	\centering
	\includegraphics[width=0.4\linewidth]{img/registração}
	\caption{legenda aqui}
	\label{fig:registracao}
\end{figure}

\begin{figure}[tbh]
	\centering
	\includegraphics[width=0.4\linewidth]{img/eyefish}
	\caption{legenda aqui}
	\label{fig:eyefish}
\end{figure}

\begin{figure}[tbh]
	\centering
	\includegraphics[width=0.4\linewidth]{img/esfera}
	\caption{legenda aqui}
	\label{fig:esfera}
\end{figure}


\begin{figure}[tbh]
	\centering
	\includegraphics[width=0.7\linewidth]{"img/360 Video - building"}
	\caption{legenda aqui}
	\label{fig:building_360_video}
\end{figure}






\subsection{Projeção}
\href{https://wiki.panotools.org/Cubic_Projection}{https://wiki.panotools.org/Cubic\_Projection}

\subsubsection{Projeção Cubica Rectilinear}
\href{https://wiki.panotools.org/Panorama_Viewers}{https://wiki.panotools.org/Panorama\_Viewers}

\begin{figure}[tbh]
	\centering
	\includegraphics[width=0.7\linewidth]{"img/projecao_cmp"}
	\caption{legenda aqui}
	\label{fig:projecao_cmp}
\end{figure}

\subsubsection{Projeção Cubica RectilinearProjeção Equirretangular (Projeção cilíndrica equidistante ou projeção de Plate Carré)}

\begin{itemize}
	\item \url{https://www.infoescola.com/cartografia/projecao-cilindrica-equidistante/}
	\item \url{https://en.wikipedia.org/wiki/Equirectangular_projection}
	\item \url{https://pt.wikipedia.org/wiki/Proje%C3%A7%C3%A3o_cil%C3%ADndrica}
	\item \url{https://pt.wikipedia.org/wiki/Proje%C3%A7%C3%A3o_cil%C3%ADndrica_equidistante}
	\item \url{https://docs.qgis.org/3.40/pt_BR/docs/gentle_gis_introduction/coordinate_reference_systems.html}
	\item \url{https://www.fcav.unesp.br/Home/departamentos/engenhariarural/TERESACRISTINATARLEPISSARRA/edital.pdf}
	\item \url{https://brasilescola.uol.com.br/geografia/projecoes-cartograficas.htm}
	\item \url{https://www.infoescola.com/cartografia/projecao-cilindrica-equidistante/}
	\item \url{https://proj.org/en/stable/operations/projections/eqc.html}	
\end{itemize}

Os prints abaixo são desse livro

\url{https://www.google.com.br/books/edition/Flattening_the_Earth/0UzjTJ4w9yEC?hl=pt-BR&gbpv=1&dq=isbn:0226767477&printsec=frontcover}

\begin{figure}[tbh]
	\centering
	\includegraphics[width=0.7\linewidth]{"img/screenshot_livro1"}
	\caption{legenda aqui}
	\label{fig:screenshot_livro1}
\end{figure}

\begin{figure}[tbh]
	\centering
	\includegraphics[width=0.7\linewidth]{"img/screenshot_livro2"}
	\caption{legenda aqui}
	\label{fig:screenshot_livro2}
\end{figure}

\begin{figure}[tbh]
	\centering
	\includegraphics[width=0.7\linewidth]{"img/tissot"}
	\caption{Indicador de tissot para a projeção cubemap}
	\label{fig:tissot}
\end{figure}



\subsection{Tiling}
\subsection{codificação}
\subsection{Dashing}
\subsection{Viewport}
\subsection{Limitações da Literatura}
\subsection{Conexão como trabalho proposto}
\subsection{Avaliação de Desempenho}


\section{360EAVP}
\subsection{Preparação}
\subsection{Identificador de face}
\subsection{Preditor de Viewport}
\subsection{Sobre a estrutura do programa}

\section{Modelagem de QoE}


\chapter{O dataset de métricas}\label{Cap:Dataset}

\section{O dataset de movimento de cabeça usado}
\section{Seleção de dados e sobre o SI/TI}
\section{O movimento de cabeça}
\section{A preparação do vídeo para o stream}\\
\subsection{O recorte}
\subsection{A codificação}
\subsection{Segmentação temporal e o DASH}
\subsection{Taxa de bits e a Decodificação}\\
\section{Métricas de Qualidade}
\section{ladrilhos vistos}
\section{A estrutura do Dataset}
\subsection{Métricas Ojetivas}\\
\subsection{Taxa de bits e a Decodificação}\\
\subsection{Seen tiles}\\




\chapter{Resultados da Análise}\label{Cap:Results}
\section{Análise Geral dos chunks}
\subsection{Taxa de bits}\\
\subsection{Tempo de decodificação}\\
\subsection{Qualidade}\\
\subsection{GetTiles}\\
\section{Análise dos chunks por qualidade e ladrilhamento}
\subsection{Taxa de bits}\\
\section{Análise dos chunks por ladrilhamento}
\section{Análise dos chunks por ladrilhamento e por qualidade}
\section{Análise dos ladrilhos}


%\include{08_cp5_emapproach}
%\include{09_cp6_conclusions}

%\include{11-Cap_3_FundamentacaoTeorica}
%\include{10-Cap_2_RevisaoBibliografica}
%\include{13-Cap_5a_GM}
%\include{12-Cap_4_MelhoriasPropostas}
%\include{13-Cap_5_ApresAnalDosResultados}
%\include{14-Cap_6_ConclusaoPropostas}

% DEFINE CABEÇALHO PARA A BIBLIOGRAFIA
\pagestyle{fancy}
\fancyhead[RE,RO]{\fontsize{9}{10}\textit{\textsc{\thepage}}}
\fancyhead[LE,LO]{\fontsize{9}{10}\textit{\textsc{\nouppercase{\rightmark}}}}

%% BIBLIOGRAFIA (GERADO AUTOMATICAMENTE)
\clearpage
\phantomsection
\addcontentsline{toc}{chapter}{References}
\bibliography{library}

% DEFINE CABEÇALHO PARA O APÊNDICE
\pagestyle{fancy}
\fancyhead[RE,RO]{\fontsize{9}{10}\textit{\textsc{\thepage}}}
\fancyhead[LE,LO]{\fontsize{9}{10}\textit{\textsc{\nouppercase{\rightmark}}}}

%\appendix
%\include{15-Apendices}

\end{document}