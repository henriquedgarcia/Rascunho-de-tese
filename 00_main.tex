\documentclass{unbthesis}
\usepackage{amssymb,amsfonts,amstext,amsmath}
\DeclareMathOperator{\sen}{sen}
\DeclareMathOperator{\argum}{arg}
\usepackage[english,brazil]{babel}
\usepackage{blindtext}
\usepackage[T1]{fontenc}
\usepackage[utf8]{inputenc} 
\usepackage{color}
\usepackage{subfigure}
\usepackage{setspace}
\usepackage{longtable}
\usepackage{float}
\usepackage{colortbl}
\usepackage{verbatim}
\usepackage{lscape}
\usepackage{indentfirst}
\usepackage[bookmarks=true,bookmarksopen=false,linktocpage=true,pagebackref]{hyperref}
\usepackage[alf,abnt-etal-text=it,abnt-etal-list=0,abnt-etal-cite=2,abnt-and-type=&,abnt-year-extra-label=yes]{abntex2cite}

\usepackage{algorithmic}
\usepackage{textcomp}
\usepackage{stfloats}
\usepackage{xfrac}
\usepackage{booktabs}
\usepackage{soul}
\usepackage{subfigure}
\usepackage{multirow}
\usepackage{hhline}
\usepackage{graphics,graphicx}
\usepackage{epstopdf}
\usepackage{array}
\usepackage{icomma}

% Configurações do pacote backref
% Usado sem a opção hyperpageref de backref
\renewcommand{\backrefpagesname}{Cited in page(s):~}
% Texto padrão antes do número das páginas
\renewcommand{\backref}{}
% Define os textos da citação
\renewcommand*{\backrefalt}[4]{
\ifcase #1 %
No citation in text.%
\or
Cited in page #2.%
\else
Cited #1 times in pages #2.%
\fi}%


\usepackage{booktabs}
\usepackage{ctable}
\usepackage{multirow}
\usepackage{fancyhdr}
\usepackage{fancyvrb}
\usepackage{relsize}
\usepackage{calc}
\usepackage{psfrag}
\usepackage{epstopdf}
\usepackage{wallpaper}
\usepackage{paralist}
%\usepackage{hhline} %pacote para criar linhas horizontais segmentadas
%\usepackage{url}
%\usepackage{layout}
%\usepackage[retainorgcmds]{IEEEtrantools}
\usepackage{afterpage}

\setlength{\LTcapwidth}{17cm} \setlength{\textwidth}{17cm}
\setlength{\textheight}{24.9cm} \setlength{\topmargin}{-0.5in}

%NOVOS COMANDOS
\newcommand{\FuncSen}[1]{\mbox{\hspace{0.5mm}sen\hspace{0.5mm}}#1}
\newcommand{\FuncCos}[1]{\mbox{\hspace{0.5mm}cos\hspace{0.5mm}}#1}
\newcommand{\palfa}[1]{\mbox{plano-$\alpha$}#1}

\usepackage{mathtools}
\DeclarePairedDelimiter\abs{\lvert}{\rvert}%

%DEIXA ESPAÇOS EXTRAS, QUE O LATEX CRIA, NO FIM DA PÁGINA
%\raggedbottom



\addto\captionsbrazil{%
	\renewcommand{\contentsname}{Table of Contents}
	\renewcommand{\listfigurename}{List of Figures}
	\renewcommand{\listtablename}{List of Tables}
	\renewcommand{\refname}{References}
	\renewcommand{\bibname}{References}
	\renewcommand{\chaptername}{Chapter}
	\renewcommand{\figurename}{Figure}
	\renewcommand{\tablename}{Table}
	\renewcommand{\acknowledgementsname}{Acknowledgements}
}


%%%VARIÁVEIS DO TEMPLATE - AJUSTAR DE ACORDO - se usar direito vai ter menos trabalho!

%variáveis exibidas na capa e contracapa
\newcommand{\ttitle}{Rascunho de tese: A dataset for performance assessments of ABR algorithm s on tiled 360° video streaming}
\newcommand{\tauthor}{Comissão de internacionalização PPGEE/UnB}
\newcommand{\tdoctype}{Exame de Qualificação de Tese de Doutorado}
% ou \newcommand{\tdoctype}{Tese de Doutorado}
\newcommand{\tcourse}{Engenharia Elétrica}
\newcommand{\tdepartment}{Departamento de Engenharia Elétrica}
\newcommand{\tdpt}{ENE} %ou ENC ENM etc

\newcommand{\tdegree}{Doutor}
% ou \newcommand{\tdegree}{Doutor}

%variáveis utilizadas na ficha catalográfica
\newcommand{\tlocal}{Brasília/DF}
\newcommand{\tday}{03}
\newcommand{\tmonth}{setembro}
\newcommand{\tyear}{2022}

%nome para citação
\newcommand{\tsurname}{Sobrenome}
\newcommand{\tfirstname}{Nome}



\begin{document}

%PEDIDO DE MODELO COM AS DIMENSÕES DA PÁGINA
%PRECISA DO PACKAGE "layout"
%\layout

%NOVO ESPAÇAMENTO
\baselineskip 20pt
\parskip 5pt
\pagestyle{empty}
\setlength{\headheight}{14.49998pt}

\pdfbookmark[0]{Cover}{Cover}
\include{01_cover}
\include{02_approval}
\include{03_copyright}

%NOVO ESPAÇAMENTO
\baselineskip 22pt
\pdfbookmark[0]{Dedication}{Dedication}
\include{03.1_dedicatory}
\pdfbookmark[0]{Acknowledgements}{Acknowledgements}
\acknowledgements
Agradeço à família que ajuda e atrapalha, mas nunca me abandona.
\pdfbookmark[0]{Abstract}{Abstract}
\include{03.3_abstract}
\pdfbookmark[0]{Resumo}{Resumo}
\include{03.4_resumo}
%% DEFINE CABEÇALHO DO SUMÁRIO E DAS LISTAS

%IMPRESSÃO SOMENTE FRENTE
\pagestyle{fancy}
\renewcommand{\chaptermark}[1]{\markboth{ #1}{}}
\fancyhf{}
\fancyhead[RE,RO]{\fontsize{9}{10}\thepage}
\fancyhead[LO,LE]{\fontsize{9}{10}\textit{\textsc{\nouppercase\leftmark}}}

%NOVO ESPAÇAMENTO
\baselineskip 14pt
\pagenumbering{roman}

%% Sumário (gerado automáticamente)
\phantomsection %comando para fazer os bookmarks funcionarem na seção abaixo
\addcontentsline{toc}{chapter}{Table of contents}
\tableofcontents

%NOVO ESPAÇAMENTO
\baselineskip 23pt

%% Lista de figuras (gerada automaticamente)
\clearpage
\phantomsection
\addcontentsline{toc}{chapter}{List of figures}
\listoffigures

%% Lista de tabelas (gerada automaticamente)
\clearpage
\phantomsection
\addcontentsline{toc}{chapter}{List of tables}
\listoftables

%% Lista de Simbolos (gerada manualmente)
\markboth{LIST OF SYMBOLS}{LIST OF SYMBOLS}
\clearpage
\phantomsection
\addcontentsline{toc}{chapter}{List of symbols}
\chapter*{List of Symbols}

\noindent
\pagestyle{fancy}

\begin{longtable}{l l l p{0.86\linewidth}}
$\varepsilon_r$ &   & Relative electric permittivity &[p.u.]\\\\

\end{longtable} 

%% Glossario (gerada manualmente)
\markboth{GLOSSARY}{GLOSSARY}
\clearpage
\phantomsection
\addcontentsline{toc}{chapter}{Glossary}
\include{11_glossary}

\pagestyle{empty}
\pagenumbering{arabic}

%NOVO ESPAÇAMENTO
\baselineskip 23pt
\parskip 5pt

%IMPRESSÃO SOMENTE FRENTE
\pagestyle{fancy}
\renewcommand{\chaptermark}[1]{\markboth{\thechapter\hspace{2mm}--\hspace{2mm}#1}{}}
\renewcommand{\sectionmark}[1]{\markright{\thesection\hspace{2mm}--\hspace{2mm}#1}{}}
\fancyhf{}
\fancyhead[RE,RO]{\fontsize{9}{10}\textit{\textsc{\thepage}}}
\fancyhead[LE,LO]{\fontsize{9}{10}\textit{\textsc{\rightmark}}}
\renewcommand{\footrulewidth}{0.0pt} %Definindo a espessura de uma linha do rodapé


\chapter{Introduction}\label{Cap:Introduction}
	
	\section{READ ME FIRST!}
	This template \LaTeX was prepared by the Internationalization Committee of the Post Graduate Program in Electrical Engineering (PPGEE) of the University of Brasilia, with the purpose of providing guidelines to students with dissertations or thesis works in progress.
	
	Under the internal regulations of PPGEE, Doctoral dissertations must be written in English. The master's thesis should be written in English, but Portuguese may be acceptable. Note that the use of terms `dissertation' and `thesis' in English are switched if compared to the ideas conveyed in Portuguese language. This is not an error; it is just how the English language works. If in doubt, refer to \href{https://www.enago.com/academy/thesis-vs-dissertation/}{this page}.
	
	This document has been adjusted to comply with Brazilian ABNT standards for academic writing, whilst keeping English as the default language. Observe that captions in the cover page, signatures page and catalog record page are supposed to be kept in Portuguese. The same happens with the section `Resumo'. Even if the chosen idiom is English, the Portuguese version must be provided.
	
	Citations follow the ABNT standard, for instance \cite{Saadat1999,ABNT1993}. You can use $\backslash$citeonline, as spoke \citeonline{MatthewN.O.Sadiku2000}, or maybe someone else.
	
	Table titles are placed at the top, sources at the bottom. \textbf{It is mandatory to specify the sources (references)} for all tables, even if they were compiled by the author - in such cases, a note `Own authorship' is sufficient.
		\begin{table}[H]
		\renewcommand{\arraystretch}{1.3}
		\caption{Relative permitivitty ($\varepsilon_r$) and electrical resistivity ($\rho$) of soils and common materials }
		\label{table:soil_properties}
		\centering
		\begin{tabular}{|c|c|c|c|c|c|}
			\hline
			\textbf{Dry materials} & \boldmath{$\varepsilon_r$} & \boldmath{$\rho$ [$\Omega$.m]} & 
			\textbf{Saturated materials} &
			\boldmath{$\varepsilon_r$} &
			\boldmath{$\rho$ [$\Omega$.m]} \\
			\hline
			Air & 1 & $10^9$--$10^{15}$ & Distilled water & 81 & $10^5$ \\ \hline
			Sand and gravel & 2--6 & $10^5$ & Fresh water & 81 & 2000 \\ \hline
			Clay & 5 & 300--5000 & Sea water & 81 & <10 \\ \hline
			Shale and dry silt & 5 & 1000 & Sand & 20--30 & 1000--$10^4$ \\ \hline
			Limestone gravel & 4 & $7\times10^6$ & Silt & 10 & 100--1000 \\ \hline
			Sandy soil & 2.6 & 1000--8000 & Clay & 40 & <10 \\ \hline
			Loamy soil & 2.4 & 300--5000 & Sandy soil & 25 & <150 \\ \hline
			Granite & 5 & 1500--$10^4$ & Granite & 7 & 1000 \\ \hline
			Limestone & 4 & 500--5000 & Limestone & 8 & 500 \\ \hline
			Salt & 5--6 & 1000--$10^5$ & Loamy soil & 15 & 20 \\ \hline
			Granite gravel & 5 & $1.5\times10^6$--$4.5\times10^{6}$ & Granite gravel & 7 & 5000--$10^4$ \\ \hline
			Basalt & 6 & 1000 & Silt & 30 & 10 \\ \hline
			Diabase & 7 & 100 & Shale & 7 & 10 \\ \hline
			Iron & 1 & $9.70\times10^{-8}$ & Limestone gravel & 8 & 2000--3000 \\ \hline
			Carbon steel & 1 & $1.43\times10^{-7}$ & Diabase & 8 & 10 \\ \hline 
			PVC & 8 & $15\times10^{17}$ & Basalt & 8 & 100 \\ \hline
			Asphalt & 3--5 & $2\times10^{6}$--$30\times10^{6}$ & Asphalt & 3--5 & $10^4$--$6\times10^{6}$ \\ \hline
			Dry concrete & 5.5 & $10^6$--$10^{9}$ & Wet concrete & 12.5 & 21--100 \\
			\hline
		\end{tabular}
		\vspace{0.05cm}\\
		\textbf{Source:} ~Own authorship.\\
	\end{table}
	
	The same happens with the figures: titles are placed at the top, sources at the bottom. \textbf{It is mandatory to specify the sources (references)} for all figures and graphs, even if they were compiled by the author - in such cases, a note `Own authorship' is sufficient.
	\begin{figure}[H]
		\begin{center}
		\caption{Just a sample figure.} 
			\includegraphics[width=.1\columnwidth]{./CapaUNB_Nova.eps}
				\vspace{0.05cm}\\
		\textbf{Source:} ~Own authorship.\\
				\label{fig:sample_figure}
			\end{center}
		\end{figure}

    Text encoding has been set to UTF-8. This is important to allow displaying accented characters in Portuguese without loss of compatibility with the \textit{babel} English package. Note that UTF-8 encoding is set in line 8 of the main \LaTeX file, as well as hard-coded in line 19 of the \textit{unbthesis.cls} class file. Be careful when messing around with character encoding to prevent conflicts and compiling issues.
    
    This is a constant work in progress and new improvements are expected over time. Instructions and responses to FAQs will be added to this section as need. You can contact the main author of this template via email: \href{mailto:amaurigm@lapse.unb.br}{amaurigm@lapse.unb.br}.
%
\chapter{FUNDAMENTAÇÃO TEÓRICA E TRABALHOS RELACIONADOS}\label{Cap:Foundations}

\section{Transmissão de vídeos 360 com ladrilhos}

O streaming de vídeo 360 envolve várias etapas de processamento de imagens, como mostra a figura XXXX. O processo começa com a captura do vídeo através de arranjos de câmeras tradicionais. As imagens precisam ser processadas, coladas e então projetadas em uma superfície plana onde um codificador de vídeo como H.265 ou AV1 é usado para comprimi-lo. Como o campo visual do ser humano é limitado, o usuário encherga apenas uma fração de toda as esfera. Assim, para que o que é visto tenha uma boa resolução, como Full DH, a projeção deve ter uma resolução muito maior, como 4K ou até mesmo 12K. Porém, como processar partes da esfera que não são assistidas desperdiçam recursos, a projeção é segmentada espacialmente em ladrilhos (tiles) que possam ser decodificados forma independentes.

A criação do streaming de vídeo esférico envolve várias etapas envolvendo diversas técnicas de processamento de imagem para

Captura -> stitch --> projeção --> tiling --> codificação -->  Dashing --> Streaming --> decoding --> mount --> Rectilinear Projection (Viewport) --> display

\subsection{Captura}

\begin{figure}[tbh]
	\centering
	\includegraphics[width=0.4\linewidth]{img/captura1}
	\caption{legenda aqui}
	\label{fig:captura1}
\end{figure}

\begin{figure}[tbh]
	\centering
	\includegraphics[width=0.4\linewidth]{img/registração}
	\caption{legenda aqui}
	\label{fig:registracao}
\end{figure}

\begin{figure}[tbh]
	\centering
	\includegraphics[width=0.4\linewidth]{img/eyefish}
	\caption{legenda aqui}
	\label{fig:eyefish}
\end{figure}

\begin{figure}[tbh]
	\centering
	\includegraphics[width=0.4\linewidth]{img/esfera}
	\caption{legenda aqui}
	\label{fig:esfera}
\end{figure}


\begin{figure}[tbh]
	\centering
	\includegraphics[width=0.7\linewidth]{"img/360 Video - building"}
	\caption{legenda aqui}
	\label{fig:building_360_video}
\end{figure}






\subsection{Projeção}
\href{https://wiki.panotools.org/Cubic_Projection}{https://wiki.panotools.org/Cubic\_Projection}

\subsubsection{Projeção Cubica Rectilinear}
\href{https://wiki.panotools.org/Panorama_Viewers}{https://wiki.panotools.org/Panorama\_Viewers}

\begin{figure}[tbh]
	\centering
	\includegraphics[width=0.7\linewidth]{"img/projecao_cmp"}
	\caption{legenda aqui}
	\label{fig:projecao_cmp}
\end{figure}

\subsubsection{Projeção Cubica RectilinearProjeção Equirretangular (Projeção cilíndrica equidistante ou projeção de Plate Carré)}

\begin{itemize}
	\item \url{https://www.infoescola.com/cartografia/projecao-cilindrica-equidistante/}
	\item \url{https://en.wikipedia.org/wiki/Equirectangular_projection}
	\item \url{https://pt.wikipedia.org/wiki/Proje%C3%A7%C3%A3o_cil%C3%ADndrica}
	\item \url{https://pt.wikipedia.org/wiki/Proje%C3%A7%C3%A3o_cil%C3%ADndrica_equidistante}
	\item \url{https://docs.qgis.org/3.40/pt_BR/docs/gentle_gis_introduction/coordinate_reference_systems.html}
	\item \url{https://www.fcav.unesp.br/Home/departamentos/engenhariarural/TERESACRISTINATARLEPISSARRA/edital.pdf}
	\item \url{https://brasilescola.uol.com.br/geografia/projecoes-cartograficas.htm}
	\item \url{https://www.infoescola.com/cartografia/projecao-cilindrica-equidistante/}
	\item \url{https://proj.org/en/stable/operations/projections/eqc.html}	
\end{itemize}

Os prints abaixo são desse livro

\url{https://www.google.com.br/books/edition/Flattening_the_Earth/0UzjTJ4w9yEC?hl=pt-BR&gbpv=1&dq=isbn:0226767477&printsec=frontcover}

\begin{figure}[tbh]
	\centering
	\includegraphics[width=0.7\linewidth]{"img/screenshot_livro1"}
	\caption{legenda aqui}
	\label{fig:screenshot_livro1}
\end{figure}

\begin{figure}[tbh]
	\centering
	\includegraphics[width=0.7\linewidth]{"img/screenshot_livro2"}
	\caption{legenda aqui}
	\label{fig:screenshot_livro2}
\end{figure}

\begin{figure}[tbh]
	\centering
	\includegraphics[width=0.7\linewidth]{"img/tissot"}
	\caption{Indicador de tissot para a projeção cubemap}
	\label{fig:tissot}
\end{figure}



\subsection{Tiling}
\subsection{codificação}
\subsection{Dashing}
\subsection{Viewport}
\subsection{Limitações da Literatura}
\subsection{Conexão como trabalho proposto}
\subsection{Avaliação de Desempenho}


\section{360EAVP}
\subsection{Preparação}
\subsection{Identificador de face}
\subsection{Preditor de Viewport}
\subsection{Sobre a estrutura do programa}

\section{Modelagem de QoE}

%\include{06_cp3_bibreview}
%\include{07_cp4_circuitapproach}
%\include{08_cp5_emapproach}
%\include{09_cp6_conclusions}

%\include{11-Cap_3_FundamentacaoTeorica}
%\include{10-Cap_2_RevisaoBibliografica}
%\include{13-Cap_5a_GM}
%\include{12-Cap_4_MelhoriasPropostas}
%\include{13-Cap_5_ApresAnalDosResultados}
%\include{14-Cap_6_ConclusaoPropostas}

% DEFINE CABEÇALHO PARA A BIBLIOGRAFIA
\pagestyle{fancy}
\fancyhead[RE,RO]{\fontsize{9}{10}\textit{\textsc{\thepage}}}
\fancyhead[LE,LO]{\fontsize{9}{10}\textit{\textsc{\nouppercase{\rightmark}}}}

%% BIBLIOGRAFIA (GERADO AUTOMATICAMENTE)
\clearpage
\phantomsection
\addcontentsline{toc}{chapter}{References}
\bibliography{library}

% DEFINE CABEÇALHO PARA O APÊNDICE
\pagestyle{fancy}
\fancyhead[RE,RO]{\fontsize{9}{10}\textit{\textsc{\thepage}}}
\fancyhead[LE,LO]{\fontsize{9}{10}\textit{\textsc{\nouppercase{\rightmark}}}}

%\appendix
%\include{15-Apendices}

\end{document} 